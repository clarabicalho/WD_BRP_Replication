\begin{table}[htb]
\centering \footnotesize
\caption{Covariate Adjusted Lin Estimates} 
\label{tab_lin_main}
\begin{tabular}{llccc}
\hline\hline && \multicolumn{3}{c}{Outcome} \\
Sample & Condition & Y1: Expected Behavior & Y2: Message Writing & Y3: Externality Content \\ 
& & (1) & (2) & (3) \\ \hline
 & T1 & -0.089 & -0.001 & 0.012 \\ 
  Pooled &  & (0.053) & (0.026) & (0.03) \\ 
   & T2 & 0.031 & -0.004 & 0.023 \\ 
   &  & (0.053) & (0.026) & (0.03) \\ \cline{2-5}
   & $R^2$ & 0.036 & 0.061 & 0.052 \\ 
   & Adj. $R^2$ & 0.019 & 0.044 & 0.021 \\ 
   & N & 2576 & 2568 & 1382 \\ \hline
   & T1 & -0.222* & -0.01 & 0.055 \\ 
  Kenya &  & (0.096) & (0.05) & (0.054) \\ 
   & T2 & -0.128 & -0.02 & 0.102 \\ 
   &  & (0.098) & (0.049) & (0.059) \\ \cline{2-5}
   & $R^2$ & 0.093 & 0.11 & 0.104 \\ 
   & Adj. $R^2$ & 0.037 & 0.055 & -0.006 \\ 
   & N & 604 & 603 & 323 \\ \hline
   & T1 & -0.024 & -0.029 & -0.033 \\ 
  Nigeria &  & (0.063) & (0.032) & (0.034) \\ 
   & T2 & -0.033 & 0.02 & -0.017 \\ 
   &  & (0.066) & (0.032) & (0.033) \\ \cline{2-5}
   & $R^2$ & 0.041 & 0.04 & 0.052 \\ 
   & Adj. $R^2$ & 0.016 & 0.015 & 0.001 \\ 
   & N & 1467 & 1461 & 739 \\ \hline
   & T1 & -0.019 & 0.053 & 0.036 \\ 
  Uganda &  & (0.109) & (0.051) & (0.065) \\ 
   & T2 & 0.229* & 0.007 & 0.001 \\ 
   &  & (0.106) & (0.05) & (0.064) \\ \cline{2-5}
   & $R^2$ & 0.115 & 0.142 & 0.127 \\ 
   & Adj. $R^2$ & 0.049 & 0.078 & 0.02 \\ 
   & N & 505 & 504 & 320 \\ 
   \hline
\end{tabular}
\begin{flushleft}\textit{Note:} We display coefficients from linear models interacting centered covariates with treatment conditions (see Lin (2013) for details on estimation method). Coefficients therefore refer to the adjusted average treatment effect of each of the two treatment conditions: social pressure (T1) and material cost (T2). Covariates include age, gender, education level, religiosity, occupation, self-reported urban/rural location, a dummy for whether the respondent voted for the incumbent in previous election or is copartisan with incumbent's party, a dummy for whether respondent under lockdown policy. Country rows refer to regression coefficients from country samples. The pooled regression includes country fixed effects and observations are reweighted such that countries are weighted equally (rather than individuals being weighted equally). This avoids that results are driven primarily by Nigerian respondents, which comprise a majority (56 percent) of our sample.  Column (1) outcome refers to the scaled answer to the vignette experiment for respondent beliefs about how likely the hypothetical individual is to accept a dinner invitation from a friend (1-very unlikely, 5 - very likely). Column (2) outcome is whether the respondent wrote a message encouraging others to practice physical distancing.  Column (3) outcome refers to whether the content of the message written appealed to coordinated collective effort (e.g., included terms such as `collaboration`, `together`, `union`). $^{*}$ $p<0.05$; $^{**}$ $p<0.01$; $^{***}$ $p<0.001$  \end{flushleft}
\end{table}
