\begin{table}[htb]
\centering
\caption{Difference in Means Estimates for T1 (Social Pressure)} 
\label{tab_dim_soc}
\begin{tabular}{lccc}
\hline\hline & \multicolumn{3}{c}{Outcome} \\
Sample & Y1: Expected Behavior & Y2: Message Writing & Y3: Externality Content \\ 
& (1) & (2) & (3) \\ \hline
Pooled & -0.098 & -0.005 & 0.013 \\ 
   & (0.074) & (0.026) & (0.03) \\ 
  Kenya & -0.308* & -0.308* & -0.562** \\ 
   & (0.127) & (0.127) & (0.174) \\ 
  Nigeria & -0.019 & -0.015 & 0.132 \\ 
   & (0.084) & (0.084) & (0.122) \\ 
  Uganda & 0.032 & 0.035 & -0.07 \\ 
   & (0.161) & (0.161) & (0.204) \\ 
   \hline
\end{tabular}
\begin{flushleft}\textit{Note:} We display difference-in-means estimates between social pressure (T1) and the control condition in the vignette experiment. Country rows refer to average treatment effects of country samples. The pooled results report a difference in means with weights adjusted such that countries are weighted equally (rather than individuals being weighted equally). This avoids that results are driven primarily by Nigerian respondents, which comprise a majority (56 percent) of our sample. Column (1) outcome refers to the scaled answer to the vignette experiment for respondent beliefs about how likely the hypothetical individual is to accept a dinner invitation from a friend (1-very unlikely, 5 - very likely). Column (2) outcome is whether the respondent wrote a message encouraging others to practice physical distancing.  Column (3) outcome refers to whether the content of the message written appealed to coordinated collective effort (e.g., included terms such as `collaboration`, `together`, `union`). $^{*}$ $p<0.05$; $^{**}$ $p<0.01$; $^{***}$ $p<0.001$  \end{flushleft}
\end{table}
